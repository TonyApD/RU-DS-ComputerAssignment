\documentclass{article}
\usepackage[utf8]{inputenc}
\usepackage{array}
\usepackage{multicol}
\usepackage{listings}
\usepackage{amssymb}
\usepackage{enumitem}
\usepackage{graphicx}
\usepackage{amsthm}
\usepackage{hyperref}
\usepackage{tikz}
\usepackage{amssymb}
\usepackage{amsmath}
\usepackage{mathtools}

\begin{document}
\title{Discrete Structures \\ Computer Assignment}
\date{\today}
\author{Tony Lopar s1013792 \\ Carlo Jessurun s1013793 \\ Marnix Dessing s1014097}
\maketitle

\section{Language}
The language we used is Python. This is the only sensible choice based on the fact the framework given for the assignment is written in Python. And due to the time frame we don't have time to write our own framework. Also Python has a calculus library Sympy (see next section). Since we are allowed to use Simpy, it was the most obvious choice to work with that.

\section{Libraries}
We use the following libraries:

\textbf{1.}: The framework given by the course.

\textbf{2.}: Sympy

\subsection{Framework}
The framework pretty much explains its self.
We had to implement the parsing of the F(n) our self, after the associated homogeneous part was parsed. Broadly speaking we use the following approach in our solution:

\textbf{Step 1}: Rewrite in the default form

\textbf{Step 2}: Determine characteristic equation

\textbf{Step 3}: Find roots and multiplicities of characteristic equation

\textbf{Step 4}: Write down general solution

\textbf{Step 5}: Use initial conditions to determine values of the parameters

\newpage
\section{Problems and solutions}
We had some problems with inhomogeneous relations. We did however find solutions to most of the homogeneous relations given by the test. Working further on solving inhomogeneous relations we did encounter a lot of problems.
\newline
\newline
We had one interesting problem for testdata: ``comass6.txt''. For some reason our program seemed to be looping on that one without printing an error or solving anything. We continued without solving that solution and commented it out because we did not want it to halt our program. We added the output from ``comass6.txt'' down below.
\newline
\newline
The problem with the inhomogeneous relations was mainly situated around the fact that we did not know where to start with solving this problem.
We do however think that we solved enough of the homogeneous ones and are happy with our result this far.
\newline
\newline
``Gerneral solution: a0*(1)**n + a1*(-3)**n + a2*(-sqrt(85) - 2)**n + a3*(-2 + sqrt(85))**n + a4*(-sqrt(7) + 1)**n + a5*n*(-sqrt(7) + 1)**n + a6*n**2*(-sqrt(7) + 1)**n + a7*(1 + sqrt(7))**n + a8*n*(1 + sqrt(7)
)**n + a9*n**2*(1 + sqrt(7))**n
''

\end{document}
