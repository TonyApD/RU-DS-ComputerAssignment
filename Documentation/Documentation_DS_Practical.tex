\documentclass{article}
\usepackage[utf8]{inputenc}
\usepackage{array}
\usepackage{multicol}
\usepackage{listings}
\usepackage{amssymb}
\usepackage{enumitem}
\usepackage{graphicx}
\usepackage{amsthm}
\usepackage{hyperref}
\usepackage{tikz}
\usepackage{amssymb}
\usepackage{amsmath}
\usepackage{mathtools}

\begin{document}
\title{Discrete Structures \\ Computer Assignment}
\date{\today}
\author{Tony Lopar s1013792 \\ Carlo Jessurun s1013793 \\ Marnix Dessing s1014097}
\maketitle

\section{Language}
The language we used is Python. This is the only sensible choice based on the fact the framework given for the assignment is written in Python. And due to the time frame we don't have time to write our own framework. Also Python has a calculus library Sympy (see next section). Since we are allowed to use Simpy, it was the most obvious choice to work with that.

\section{Libraries}
We use the following libraries:
\begin{enumerate}
    \item The framework that comes with the assignment.
    \item Sympy
\end{enumerate}

\subsection{Framework}
The framework pretty much explains its self.
We had to implement the parsing of the F(n) our self, after the associated homogeneous part was parsed. We had the following approach: tony jij hebt dit gedaan toch?

\subsection{Sympy}
Homogeneous\\

We now have a direct solution for our recurrence relation.

\section{Problems and solutions}
Problem 1:

Solving a homogeneous relation for some init values was impossible...


\end{document}
